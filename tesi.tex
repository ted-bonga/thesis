% Classe appositamente creata per tesi di Ingegneria Informatica all'università Roma Tre
\documentclass{TesiDiaUniroma3}

% --- INIZIO dati relativi al template TesiDiaUniroma3
% dati obbligatori, necessari al frontespizio
%%\titolo{Questo \`e il titolo della tesi}
\titolo{Progettazione e sviluppo di un'interfaccia web in Laravel e Vue.js per il monitoraggio interattivo e la gestione di stazioni di ricarica per veicoli elettrici}
\autore{Ted Jevel Quilenderino Bonga}
\matricola{546569}
\relatore{Prof. Paolo Merialdo}
\correlatore{Prof.sa Silvia Canale} % modifica anche TesiDiaUniroma3.cls se vuoi avere un correlatore
\annoAccademico{2023-2024}

 % dati opzionali
\dedica{Questa \`e la dedica} % solo se nel documento si usa il comando \generaDedica
% --- FINE dati relativi al template TesiDiaUniroma3

% --- INIZIO richiamo di pacchetti di utilità. Questi sono un esempio e non sono strettamente necessari al modello per la tesi.
\usepackage[plainpages=false]{hyperref}	% generazione di collegamenti ipertestuali su indice e riferimenti
\usepackage[all]{hypcap} % per far si che i link ipertestuali alle immagini puntino all'inizio delle stesse e non alla didascalia sottostante
\usepackage{amsthm}	% per definizioni e teoremi
\usepackage{amsmath}	% per ``cases'' environment
\usepackage[utf8]{inputenc}
\usepackage{graphicx}
\usepackage{xcolor}
%\usepackage[backend=biber,style=alphabetic]{biblatex} 

% --- FINE riachiamo di pacchetti di utilità

\begin{document}
% ----- Pagine di fronespizio, numerate in romano (i,ii,iii,iv...) (obbligatorio)
\frontmatter
\generaFrontespizio
\generaDedica
\ringraziamenti{ringraziamenti}	% inserisce i ringraziamenti e li prende in questo caso da ringraziamenti.tex
\terminologia{terminologia}		% inserisce l'introduzione e la prende in questo caso da introduzione.tex
%\generaIndice
%\generaIndiceFigure

% ----- Pagine di tesi, numerate in arabo (1,2,3,4,...) (obbligatorio)
\mainmatter
% il comando ``capitolo'' ha come parametri:
% 1) il titolo del capitolo
% 2) il nome del file tex (senza estensione) che contiene il capitolo. Tale nome \`e usato anche come label del capitolo\documentclass{article}

\title{Progettazione e sviluppo di un'interfaccia web in Laravel e Vue.js per il monitoraggio interattivo e la gestione di stazioni di ricarica per veicoli elettrici}
\author{Ted Jevel Quilenderino Bonga}
\date {Ottobre 2024}
%%\begin{document}
\tableofcontents
\maketitle 
\newpage
\capitolo{Introduzione}{introduzione}
\capitolo{Tecnologie utilizzate}{capitolo1}
\capitolo{Architettura di riferimento}{capitolo2}
\capitolo{Analisi dei Requisiti}{capitolo3}
\capitolo{Progettazione dell'Interfaccia Utente}{capitolo4}
\capitolo{Implementazione della Soluzione}{capitolo5}
\capitolo{Conclusioni e sviluppi futuri}{capitolo6}
\cite{ocpi}

\bibliography{bibliografia}


\end{document}