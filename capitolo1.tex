\section*{Strumenti Utilizzati}
In questo capitolo verranno analizzati gli strumenti principali utilizzati per lo sviluppo del Portale web, con particolare attenzione ai framework Laravel 8 e Vue.js.

\subsection{Laravel 8}
Laravel 8 è un framework PHP moderno ed efficiente per lo sviluppo di applicazioni web. La sua architettura MVC (Model-View-Controller) consente una gestione strutturata e scalabile del codice, facilitando la separazione tra logica di business e presentazione dell’interfaccia utente. Grazie a un ampio ecosistema di pacchetti e una community attiva, Laravel offre numerose funzionalità integrate, tra cui gestione delle rotte, autenticazione, accesso ai database e molto altro, permettendo di accelerare il processo di sviluppo.

\subsection{Vue.js}
Vue.js è un framework JavaScript progressivo per la costruzione di interfacce utente interattive e reattive. Grazie al binding bidirezionale dei dati (two-way data binding) e alla gestione del DOM virtuale, Vue.js è particolarmente adatto per lo sviluppo di single-page applications (SPA). Integrabile con facilità in progetti esistenti e altamente modulare, Vue.js supporta componenti riutilizzabili e reattività avanzata, rendendolo ideale per applicazioni web dinamiche e scalabili.

\subsection{Stazione di Ricarica Alfen}
Le stazioni di ricarica Alfen sono soluzioni avanzate per la ricarica di veicoli elettrici, note per la loro affidabilità e versatilità. Queste stazioni offrono funzionalità di monitoraggio in tempo reale e sono dotate di sistemi di sicurezza integrati per garantire la protezione durante la ricarica. Grazie alla compatibilità con piattaforme di gestione come quelle basate su Laravel e Vue.js, le stazioni Alfen consentono una gestione semplificata e personalizzata, rendendole ideali per infrastrutture di ricarica scalabili e efficienti.\\

\begin{figure}[h]
    \centering
    \includegraphics[width=12cm, height=10cm]{image2.jpeg}
    \caption{Colonnina Alfen}
    \label{fig:collonina_alfen} 
\end{figure}

\subsection{Pusher}
Pusher è una piattaforma di messaggistica in tempo reale che semplifica l'implementazione di WebSocket nelle applicazioni web. Consente agli sviluppatori di creare facilmente funzionalità interattive, come chat in tempo reale e aggiornamenti dinamici, senza la necessità di gestire direttamente le complessità associate alla gestione delle connessioni WebSocket. Pusher offre un'interfaccia API intuitiva e scalabile, facilitando la comunicazione bidirezionale tra client e server. Nel contesto del nostro progetto, Pusher è stato utilizzato per implementare aggiornamenti in tempo reale degli stati delle charge-point, migliorando così l'esperienza utente e garantendo che le informazioni siano sempre aggiornate senza necessità di ricaricare la pagina.
\cite{ocpi}, \cite{laravel}, \cite{vuejs}, etc.
