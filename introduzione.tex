Il tema di questa tesi riguarda lo sviluppo e aggiornamento di un applicazione Web per la gestione di una stazione di ricarica per veicoli elettrici (EV). In particolare, tale applicazione (alla quale d'ora in avanti si farà riferimento con "Portale" presenta due scopi principali:
\begin{itemize}
	\item Monitoraggio in tempo reale delle stazioni di ricarica: Il Portale permette di monitorare lo 	stato delle stazioni di ricarica in tempo reale, visualizzando informazioni dettagliate su ogni 		punto di ricarica, come la disponibilità (Available), la preparazione (Preparing), la ricarica in 		corso (Charging), eventuali errori (Faulted), e altri stati operativi definiti dal protocollo OCPP. Questo consente agli 			operatori di avere una visione completa e aggiornata della situazione, migliorando l'efficienza 		nella gestione delle risorse e nel supporto agli utenti.
	\item Aggiornamenti reattivi senza necessità di refresh della pagina web: 
		\begin{itemize}
		\item Una delle principali caratteristiche del Portale è la sua reattività. Ogni volta che un 			Charge Point cambia stato, questa modifica viene immediatamente riflessa nel portale senza 				dover ricaricare la pagina. Questa funzionalità è resa possibile grazie all’utilizzo di 				WebSocket, che permettono di ricevere aggiornamenti in tempo reale e migliorano l’esperienza 			utente, rendendo l’applicazione più dinamica e interattiva.	
	\end{itemize}			 
	 
\end{itemize}