\section{Introduzione}
\subsection{Cos'è un sistema operativo?}
Un sistema operativo è un \underline{software} che: \begin{itemize}
\item \textbf{controlla} l'esecuzione di altri programmi
\item agisce come \textbf{interfaccia} tra applicazioni e hardware
\end{itemize}

\subsection{Prospettiva Storica}
\subparagraph{40-s mid 50s: Serial Processing}
\begin{itemize}
\item Nessun Sistema Operativo
\item Il programma interagisce direttamente con l'hardware
\item L'utente è responsabile di tutte le fasi necessaria ad avviare l'esecuzione:

\begin{itemize}
\item caricamento del compilatore e del sorgente
\item compilazione e salvataggio del compilato
\item caricamento e collegamento con moduli predefiniti
\item caricamento ed avvio del programma
\end{itemize}

\item Nessun supporto allo scheduling

\begin{itemize}
\item un job alla volta
\item prenotazione su fogli cartacei
\end{itemize}

\end{itemize}

\subparagraph{mid 50s- late 60s: Batch System}
\begin{itemize}
\item Introduzione di un software chiamato \textbf{monitor} $\rightarrow$ assimilabile ad un primo sistema operativo
\item L'utente sottomette i \textbf{job} ad un operatore $\rightarrow$ utenti non interagiscono più con l'hardware
\item L'operatore prepara un \textbf{batch} (sequenza) di job memorizzati su un dispositivo di input
\item Il monitor carica da dispositivo di input in memoria ed avvia il programma
\item Il programma \begin{itemize}
\item carica moduli necessari alla sua esecuzione tramite comandi espressi in \textbf{job control language}
\item cede il controllo al monitor in caso di interazione con dispositivi,terminazione o errore
\end{itemize}
\end{itemize}
Il monitor:\begin{itemize}
\item è responsabile di avviare il successivo job nel batch
\item risiede (almeno in parte) sempre in memoria
\end{itemize}
L'hardware insegue e risolve le nuove problematiche introdotte dall'uso del monitor: \begin{itemize}
\item impedire ai job di corrompere l'area di memoria del monitor $\rightarrow$ \textbf{Memory Protection}
\item impedire ai job di monopolizzare l'uso del sistema $\rightarrow$ \textbf{timer}
\item impedire ai job di avere accesso incontrollato ai dispositivi di I/O $\rightarrow$ \textbf{Istruzioni privilegiate}: che possono essere: \begin{itemize}
\item user-mode
\item kernel-mode
\end{itemize}
\end{itemize}
Il batch system risolve le principali cause di sottoutilizzo di \textbf{costoso} hardware nei sistemi seriali
\begin{itemize}
\item Benefici 
 \begin{itemize}
 \item ridotti i tempi di setup
 \item ridotti tempi di inattività del sistema
 \end{itemize}
\item Costi
 \begin{itemize}
 \item parte della memoria è allocata per il monitor e non può essere allocata per dai job
 \item parte del tempo del processore è consumato dal monitor
 \end{itemize}
%Slide 8 screen%
\end{itemize}
Il monitor inoltre: \begin{itemize}
\item è in grado di cedere il controllo da un job all'altro
\item maschera le latenze di I/O con l'esecuzione di altri job
\item necessita di mantenere in memoria lo stato di più job
\end{itemize}
Batch-MultiProgramming: 
\begin{itemize}
\item L'hardware ha un ruolo fondamentale nell'avanzamento dei sistemi operativi:
 \begin{itemize}
 \item \textbf{Direct Memory Access} (trasferimenti in memoria non richiedono l'uso della CPU)
 \item \textbf{Interrupt-driven I/O} (permette al processore di essere avvisato riguardo l'avanzamento delle operazioni di I/O)
 \end{itemize}
\item  Multiprogramming OS sono più complessi:
 \begin{itemize}
 \item Memory Management
 \item Politiche di scheduling più avanzate
 \end{itemize}
\item Limiti:
 \begin{itemize}
 \item Alcune classi di job possono essere svantaggiate
 \item Impossibile supportare applicazioni interattive
 \end{itemize}
\end{itemize}
\subparagraph{late 60s-present Time-Sharing System}
\begin{itemize}
\item Piuttosto che massimizzare l'utilizzo del processore si tenta di minimizzare il tempo di risposta
\item Il tempo di CPU viene preassegnato ai job secondo un certo algoritmo di \textbf{scheduling} (e.g round-robin)
\item I job possono essere interrotti anche se non hanno fatto richieste di I/O (e.g allo scadere del quanto di tempo assegnatogli)
\item Condivisione delle risorse \textbf{trasparente}:
 \begin{itemize}
 \item I job non sono consci di essere interrotti o riesumati
 \item I job non sanno di condividere risorse con altri job (e.g processore o dispositivi di I/O)
 \end{itemize}
\end{itemize}
\subsection{Sistemi operativi moderni}
Molteplici linee di evoluzione:
\begin{itemize}
\item (soft) real-time
\item multicore
\item Energy awarness
\end{itemize}
\subparagraph{Real-time}
\begin{itemize}
\item Il SO deve garantire che determinate attività completino entro una specifica scadenza temporale (deadline)
\item Hard real-time:
 \begin{itemize}
 \item La deadline \textbf{deve} essere rispettata
 \item Utile in sistemi vitali e/o industriali
 \item Tipicamente non supportato da sistemi di timesharing
 \end{itemize}
\item Soft real-time:
 \begin{itemize}
 \item La deadline \textbf{dovrebbe} essere rispettata
 item Utile in applicazioni che richiedono specifiche latenze per determinate attività (e.g registrazione monitor e audio)
 \end{itemize}
\end{itemize}
\subsection{Interrupti-driven operating systems}
\begin{itemize}
\item I sistemi batch necessitano di un meccanismo per permettere al monitor di acquisire il controllo in determinate situazioni: 
 \begin{itemize}
 \item timer scaduti
 \item eseguire istruzioni privilegiate per conto dei job
 \item gestioni di errori software
\end{itemize}
\item \textbf{Interrupt:}
 \begin{itemize}
 \item Meccanismo per interrompere il corrente di flusso di esecuzione del processore all'occorrenza di determinati eventi ed eseguire una routine di gestione dell'interrupt (\textbf{interrupt-handler})
 \end{itemize}
\item I sistemi operativi moderni sono \textbf{interrupt-driven}
 \begin{itemize}
 \item Le interruzioni regolano le interazioni tra SO, dispositivi e programmi utente
 \end{itemize}
\end{itemize}
\subparagraph{Tipi di interruzioni}
Generata da: 
\begin{itemize}
\item Eventi asincroni $\rightarrow$ \textbf{INTERRUPT}
\item Eventi sincroni $\rightarrow$ \textbf{TRAP}
\end{itemize}
I sistemi operativi moderni: 
\begin{itemize}
\item Sono interrupt-driven I/O
\item Proteggono le risorse hardware acquisendo il controllo sotto determinate condizioni:
 \begin{itemize}
 \item Un processo monopolizza l'uso della CPU $\rightarrow$ Timer
 \item Proteggere l'accesso a determinate regioni di memoria (ad esempio quelle che contengono il codice del kernel e le relative strutture dati) $\rightarrow$ \textbf{HW detection} $\rightarrow$ \textbf{TRAP}
 \item Un processo tenta di eseguire istruzioni privilegiate $\rightarrow$ \textbf{HW detection} $\rightarrow$ \textbf{TRAP}
 \end{itemize}
\item \textbf{Istruzioni privilegiate}
 \begin{itemize}
 \item Supporto hardware per avere almeno due modi di esecuzione: \textbf{kernel} e \textbf{user}
 \end{itemize}
 \begin{itemize}
 \item Kernel mode:nessuna restrizione sulle istruzioni ammissibili
 \item User mode: alcune istruzioni sono proibite e, se eseguite, generano una trap
 \end{itemize}
\item Forniscono interfaccie ai programmi utente per accedere ai servizi del SO $\rightarrow$ \textbf{System call}
\end{itemize}
\subsection{System call}
\begin{itemize}
\item Il meccanismo delle chiamate di sistema (system call o syscall) è lo strumento messo a disposizione dal SO per accedere ed eseguire il suo codice
\item Basato su \textbf{trap}
 \begin{itemize}
 \item Il SO registra (a start up) una specifica entry per gestire le syscall nel vettore di interrupt (0x80 su Linux, e 0x2E su Windows)
 \item Il codice utente: 
  \begin{itemize}
  \item utente imposta i parametri per la syscall in specifici registri del processore
  \item scatena una trap
  \end{itemize}
 \item L'interrupt handler (kernel mode):
  \begin{itemize}
  \item assume il controllo e prende i parametri dai registri  di CPU
  \item invoca la reale syscall
  \item rilascia il controllo al programma utente 
  \end{itemize}
 \end{itemize}
\end{itemize}