%In questo capitolo parlerò degli attori principali, in particolare il Portale è designato all'utilizzo esclusivo degli operatori e non dei clienti che vogliano ricaricare la loro macchina, le funzionalità richieste

Il portale prevede l'interazione con un insieme ben definito di attori. In particolare, il portale web è progettato esclusivamente per l'uso da parte degli operatori incaricati della gestione e del monitoraggio delle stazioni di ricarica per veicoli elettrici. I clienti finali, ovvero gli automobilisti che desiderano ricaricare il proprio veicolo, non hanno accesso diretto al portale e utilizzano invece altre interfacce o strumenti per l'attivazione e il pagamento delle sessioni di ricarica (eg. HMI: Human Machine Interface).

\subsection{Tipi di utente}
Gli operatori del sistema possono essere suddivisi in diverse categorie, ciascuna con permessi e funzionalità specifiche:
\begin{itemize}
\item \textbf{Amministratori}: hanno pieno accesso al sistema, possono gestire utenti, configurare le stazioni di ricarica e monitorarie l'attività complessiva.
\item \textbf{Operatori tecnici}: si occupano del monitoriraggio delle stazioni di ricarica, intervenendo in caso di guasti o anomalie
\end{itemize}

